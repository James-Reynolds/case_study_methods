\documentclass{tufte-handout}

\title{Research methods: Is everything a case study?}

\author{Dr James Reynolds}

%\date{28 March 2010} % without \date command, current date is supplied

%\geometry{showframe} % display margins for debugging page layout

\usepackage{graphicx} % allow embedded images
  \setkeys{Gin}{width=\linewidth,totalheight=\textheight,keepaspectratio}
  \graphicspath{{graphics/}} % set of paths to search for images
\usepackage{amsmath}  % extended mathematics
\usepackage{booktabs} % book-quality tables
\usepackage{units}    % non-stacked fractions and better unit spacing
\usepackage{multicol} % multiple column layout facilities
\usepackage{lipsum}   % filler text
\usepackage{fancyvrb} % extended verbatim environments
  \fvset{fontsize=\normalsize}% default font size for fancy-verbatim environments

% Standardize command font styles and environments
\newcommand{\doccmd}[1]{\texttt{\textbackslash#1}}% command name -- adds backslash automatically
\newcommand{\docopt}[1]{\ensuremath{\langle}\textrm{\textit{#1}}\ensuremath{\rangle}}% optional command argument
\newcommand{\docarg}[1]{\textrm{\textit{#1}}}% (required) command argument
\newcommand{\docenv}[1]{\textsf{#1}}% environment name
\newcommand{\docpkg}[1]{\texttt{#1}}% package name
\newcommand{\doccls}[1]{\texttt{#1}}% document class name
\newcommand{\docclsopt}[1]{\texttt{#1}}% document class option name
\newenvironment{docspec}{\begin{quote}\noindent}{\end{quote}}% command specification environment

\begin{document}

\maketitle% this prints the handout title, author, and date

\begin{abstract}
\noindent
"...while theory building from cases is increasingly prominent, challenges in writing publishable manuscripts using this research strategy exist.  Some reviewers ... may misunderstand the method (e.g., expect random sampling), or simply regard their own methods as superior"\cite{Eisenhardt2007TBfC}.
\end{abstract}

%\printclassoptions

\section{Introduction}
This two-pager 
discusses case research methodology, 
Case study research involves 
examining a small number of cases 
in great detail, 
while at the same time 
seeking findings that are generalisable\cite{Yin2009aa, Barrat2011aa, Ketokivi2014aa,}. 
Why you might use this approach 
 is discussed next, 
 followed by 
an outline of 
 two key issues in case research: 
 the duality criterion 
 and the sampling approach. 
 The two-pager closes with 
 commentary about 
 whether "everything" is really just a case study anyway, 
 and consequences for our research.

\section{Why use case research methodology?} 
Case research is particularly useful for "how" and "why" questions 
about current events, 
but where the researcher(s) cannot 
control what happens (ruling out an experimental case-control approach). 

\begin{table*}[ht]
  \centering
  \fontfamily{ppl}\selectfont
  \begin{tabular}{llll}
    \toprule
    Method & Research question & Control of events? & Contemporary events? \\
    \midrule
    Experiment & how, why? & Yes & Yes \\
    Survey & who, what, where, how many, how much? & No & Yes \\
    Archival analysis & who, what, where, how many, how much? & No & Yes/No \\
    History & how, why? & No & No \\
    Case study & how, why? & No & Yes \\
    \bottomrule
  \end{tabular}
  \caption{When to use each research method. Source: \citet{Yin2009aa}.}
  \label{tab:when_where}
  %\zsavepos{pos:normaltab}
\end{table*}

It can be useful in building new theory 
or testing existing theory, 
as even a single case might be enough to disprove 
a hypothesis\footnote{
This red horse disproves the hypothesis that all of the horses are black or brown.}.
Llooking at only a small number of cases 
allows great depth of research so as to better understand 
the complexities, causal links and reasons behind a phenomenon, 
in contrast to random sampling methods where many cases are examined, but only shallowly\cite{Denscombe2007aa}\footnote{Correlation is not causality!}.

Case research, however, is often considered inferior, including because of: confusion with other methods or the use of case studies for non-research purposes\footnote{e.g. for teaching or in journalism}; misconceptions that the cases themselves have to be representative\footnote{A critical case (that might be non-representative) might be selected to explore why something happened (e.g. Why is this horse red, but all the others are black or brown? Did someone paint it red?).} or that statistical significance is the only way to achieve scientific rigor\cite{Bonoma1984aa,}\marginnote{Eisenhardt \& Graebner (2007); Denscombe (2007), Yin (2009).}.


\subsection{The "duality criterion"}
Case research needs to be grounded in the cases' context\footnote{Allowing cases to be examined at great depth, so as to understand how and why something is the way it is.}, while at the same time seeking generalised findings applicable to cases not included in the study\footnote{i.e. making/seeking findings than can be confidently assumed to apply to some/many/most/all fields of horses.}\footnote{Ketokivi and Choi (2014)}. Denscombe (2007) suggests providing "an explicit defence against the allegation that you cannot generalize from (the) case study findings"\footnote{i.e. I checked that the red horse in my field was not a brown horse painted red, so therefore we can confidently generalize that red horses do actually exist and that they may be found in some fields.} so as to formally address this issue.

\subsection{Sampling approaches}
The case research methodology literature describes theoretical reasons to include a case in your sample, including because it is a: leading example; critical case; particularly revelatory case; representative case; case that can be studied over time; particularly unusual case; or, together with other selected cases, provides an opposite result, similar results (for replication) or represents polar extremes. Non-theoretical reasons for including a case might be convenience (especially if geographically proximate), a forced choice (e.g. because of study funding source) or opportunistic (e.g. data access)\cite{Eisenhardt1989aa, Meredith1998aa, Stuart2002aa, Voss2002aa,}\marginnote{Barratt et al. (2011), Denscombe (2007), Ketokivi and Choi (2014), Yin (2009)}. Such non-theoretical reasons are ok, but should be acknowledged and addressed.  As well, a case included for non-theoretical reasons might also be justifiable on theoretical grounds.  

How many cases?  Only a few might not provide sufficient breadth.  Too many and getting the depth of research necessary to fully understand the context and mechanisms may be impossible. Eisenhardt (1989) suggests four to ten, Yin (2009) six to ten including two to three  chosen for replication and the others chosen to explore differences. One might be enough, especially for theory testing\footnote{This one red horse disproves the theory that all horses are brown or black.  No, there is no need for me to find another red horse, one is enough. Also, I looked closely at its hair, did genetic analysis and all sorts of other things and now understand HOW a horse might be red.}.


\section{Is everything in transport research a case study?}
Most transport research is grounded in the context of a city, country or other unit of analysis. Even experimental transport modelling might make assumptions about local traffic laws, cultural practices with respect to pedestrian behaviour or queuing, or other elements grounded in the context of a case.  A \emph{key consequence for us} therefore is to recognise the need to seek generalisable findings\footnote{Are the findings of this survey applicable elsewhere?}, that can be confidentially applied to other places and cases.  The case research methodology literature also includes recommendations with respect to: framing study questions, writing case study protocols, case reports and cross-case comparisons to ensure and demonstrate academic rigor and help to direct and formalise the study; and a body of theory that supports use of case studies as a research tool.  


\nobibliography{case_study_methods.bib}
\bibliographystyle{plainnat}



\end{document}
