\documentclass{tufte-handout}

\title{Research methods: case studies}

\author{Dr James Reynolds}

%\date{28 March 2010} % without \date command, current date is supplied

%\geometry{showframe} % display margins for debugging page layout

\usepackage{graphicx} % allow embedded images
  \setkeys{Gin}{width=\linewidth,totalheight=\textheight,keepaspectratio}
  \graphicspath{{graphics/}} % set of paths to search for images
\usepackage{amsmath}  % extended mathematics
\usepackage{booktabs} % book-quality tables
\usepackage{units}    % non-stacked fractions and better unit spacing
\usepackage{multicol} % multiple column layout facilities
\usepackage{lipsum}   % filler text
\usepackage{fancyvrb} % extended verbatim environments
  \fvset{fontsize=\normalsize}% default font size for fancy-verbatim environments

% Standardize command font styles and environments
\newcommand{\doccmd}[1]{\texttt{\textbackslash#1}}% command name -- adds backslash automatically
\newcommand{\docopt}[1]{\ensuremath{\langle}\textrm{\textit{#1}}\ensuremath{\rangle}}% optional command argument
\newcommand{\docarg}[1]{\textrm{\textit{#1}}}% (required) command argument
\newcommand{\docenv}[1]{\textsf{#1}}% environment name
\newcommand{\docpkg}[1]{\texttt{#1}}% package name
\newcommand{\doccls}[1]{\texttt{#1}}% document class name
\newcommand{\docclsopt}[1]{\texttt{#1}}% document class option name
\newenvironment{docspec}{\begin{quote}\noindent}{\end{quote}}% command specification environment

\begin{document}

\maketitle% this prints the handout title, author, and date

\begin{abstract}
\noindent
"...while theory building from cases is increasingly prominent, challenges in writing publishable manuscripts using this research strategy exist.  Some reviewers ... may misunderstand the method (e.g., expect random sampling), or simply regard their own methods as superior"\cite{Eisenhardt2007TBfC}.
\end{abstract}

%\printclassoptions

This two-pager 
discusses case research methodology, 
which involves 
examining a small number of cases 
in great detail, 
but seeking generalisable findings\cite{Yin2009aa, Barrat2011aa, Ketokivi2014aa}. 
Why you might use this approach 
 is discussed next, 
 followed by 
 an outline of 
 two issues in case research: 
 the duality criterion; 
 and sampling. 
 The two-pager closes with 
 commentary about 
 whether "almost everything" in transport is really just a case study anyway, 
 and consequences for our research.

\section{Why use case research methodology?} 

Yin [2009] provides guidance on when to use each research method:

\begin{table*}[ht]
  \centering
  \fontfamily{ppl}\selectfont
  \begin{tabular}{llll}
    \toprule
    Method & Research question & Control of events? & Contemporary events? \\
    \midrule
    Experiment & how, why? & Yes & Yes \\
    Survey & who, what, where, how many, how much? & No & Yes \\
    Archival analysis & who, what, where, how many, how much? & No & Yes/No \\
    History & how, why? & No & No \\
    Case study & how, why? & No & Yes \\
    \bottomrule
  \end{tabular}
  \caption{When to use each research method. Source: \citet{Yin2009aa}.}
  \label{tab:when_where}
  %\zsavepos{pos:normaltab}
\end{table*}

Case studies are useful for "how" and "why" questions 
about current events, 
but where the researcher(s) cannot 
control what happens. 
They can be used for building new theory 
or testing existing theory. 
Even a single case might be enough to disprove 
a hypothesis\footnote{
E.g. this red horse disproves the hypothesis that all horses are black or brown.}.
Looking at only a few cases 
allows time for greater depth, 
which might help in understanding
the complexities and causal links surrounding a phenomenon.  
This contrasts to random sampling methods, where many cases are examined, but only shallowly\cite{Denscombe2007aa}\footnote{Correlation is not causality!}.

Some have claimed that case research is an inferior approach, including because of: confusion with other methods or the use of case studies for non-research purposes\footnote{e.g. for teaching or in journalism}; misconceptions that the cases themselves have to be representative\footnote{A critical case (that might be non-representative) might be selected to explore why something happened (e.g. Why is this horse red, but all the others are black or brown? Did someone paint it red?).} or that statistical significance is the only way to achieve scientific rigor. However, there is a rich body of methodological literature that explains how and why case research approaches can be used to generate defensible findings\cite{Bonoma1984aa,}\marginnote{Eisenhardt \& Graebner (2007); Denscombe (2007), Yin (2009).}. 


\subsection{The "duality criterion"...}
...relates to how case research needs to be grounded in the context of the cases being studied\footnote{Allowing cases to be examined at great depth, so as to understand how and why something is the way it is. For example, an in depth examination of this red horse, including genetic analysis etc.}, while at the same time seeking findings applicable more widely\footnote{Ketokivi and Choi (2014)}. Denscombe (2007) suggests providing "an explicit defence against the allegation that you cannot generalize from (the) case study findings"\footnote{i.e. I checked that the red horse in my field was not a brown horse painted red, so therefore we can confidently generalise that red horses do actually exist and that they may be found in some fields.} to formally address this issue. For us, this might involve answering "why is what transport is like in Melbourne, with respect to what being studied, going to tell us about what the topic of the research is like elsewhere?"

\subsection{Sampling approaches}
The case research methodology literature describes theoretical reasons to include a case in your sample\footnote{Including because it is: a leading example; a critical case; a particularly revelatory case; a representative case; a case that can be studied over time; a particularly unusual case; or, together with other selected cases, a case that provides an opposite result, similar results (for replication) or represents a polar extreme.}. Non-theoretical reasons for including a case might be convenience (especially geographical proximity), a forced choice (e.g. because of a study funding source) or opportunistic (e.g. data access)\cite{Eisenhardt1989aa, Meredith1998aa, Stuart2002aa, Voss2002aa,}\marginnote{Barratt et al. (2011), Denscombe (2007), Ketokivi and Choi (2014), Yin (2009)}. Such non-theoretical reasons are ok, but should be acknowledged and addressed, and you would want to try and understand how the cases' inclusion might (also) be justifiable on theoretical grounds.  

How many cases?  Enough to get sufficient breadth.  Not so many that you don't have time to get the necessary depth. Eisenhardt (1989) suggests four to ten, Yin (2009) six to ten including two to three chosen for replication and the others chosen to explore differences. One might be enough, especially for theory testing\footnote{This one red horse disproves the theory that all horses are brown or black.  No, there is no need for me to find another red horse, one is enough. Also, I looked closely at its hair, did genetic analysis and all sorts of other things and now understand HOW a horse might be red.}.


\section{Is everything in transport research a case study?}
Most transport research is grounded in the context of a city, country or other unit of analysis. Even transport modelling might make assumptions about local traffic laws, cultural practices with respect to pedestrian behaviour or queuing, or other elements that are typical of some, but not all, places.  A \emph{consequence for us} therefore is to recognise the need to seek generalisable findings, and to understand the extent to which (and how) our conclusions apply elsewhere\footnote{For example, a survey of Melbourne cyclists is grounded in the context of mandatory helmet laws, an automobile-dominated transport system  and a cycling culture that includes many Middle Aged Men In Lycra (MAMILS). To what extent are the findings applicable to places without mandatory helmets and/or where bicycles are used primarily for transport?}


\section{In closing...} 
...the literature includes recommendations about: framing study questions; protocols; case reports and cross-case comparisons. These can help demonstrate academic rigor, and direct and formalise a study. There is a rich body of theory that supports case research. Chances are you are already studying specific cases, so I suggest explicitly framing it as such so your work is (even) more defensible.


\nobibliography{case_study_methods.bib}
\bibliographystyle{plainnat}



\end{document}
