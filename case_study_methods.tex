\documentclass{tufte-handout}

\title{Research methods: Is everything a case study?}

\author{Dr James Reynolds}

%\date{28 March 2010} % without \date command, current date is supplied

%\geometry{showframe} % display margins for debugging page layout

\usepackage{graphicx} % allow embedded images
  \setkeys{Gin}{width=\linewidth,totalheight=\textheight,keepaspectratio}
  \graphicspath{{graphics/}} % set of paths to search for images
\usepackage{amsmath}  % extended mathematics
\usepackage{booktabs} % book-quality tables
\usepackage{units}    % non-stacked fractions and better unit spacing
\usepackage{multicol} % multiple column layout facilities
\usepackage{lipsum}   % filler text
\usepackage{fancyvrb} % extended verbatim environments
  \fvset{fontsize=\normalsize}% default font size for fancy-verbatim environments

% Standardize command font styles and environments
\newcommand{\doccmd}[1]{\texttt{\textbackslash#1}}% command name -- adds backslash automatically
\newcommand{\docopt}[1]{\ensuremath{\langle}\textrm{\textit{#1}}\ensuremath{\rangle}}% optional command argument
\newcommand{\docarg}[1]{\textrm{\textit{#1}}}% (required) command argument
\newcommand{\docenv}[1]{\textsf{#1}}% environment name
\newcommand{\docpkg}[1]{\texttt{#1}}% package name
\newcommand{\doccls}[1]{\texttt{#1}}% document class name
\newcommand{\docclsopt}[1]{\texttt{#1}}% document class option name
\newenvironment{docspec}{\begin{quote}\noindent}{\end{quote}}% command specification environment

\begin{document}

\maketitle% this prints the handout title, author, and date

\begin{abstract}
\noindent
"...while theory building from cases is increasingly prominent, challenges in writing publishable manuscripts using this research strategy exist.  Some reviewers who work on large-scale, hyopthesis-testing research may misunderstand the method (e.g., expect random sampling), or simply regard their own methods as superior"\cite{Eisenhardt2007TBfC}.
\end{abstract}

%\printclassoptions

\section{Introduction}
This is a two-pager 
discussing case study research methodologies. 
Case research involves 
examining a small number of cases 
in great detail, 
while at the same time 
seeking findings that are generalisable\cite{Yin2009aa, Barrat2011aa, Ketokivi2014aa,}. 
Why you might use this approach 
 is discussed in the next section. 
 This is followed by 
 a section outlining 
 two key issues for case research: 
 the duality criterion 
 and the sampling approach. 
 The two-pager closes with 
 commentary about 
 whether "everything" is really just a case study anyway, 
 and consequences for our research.

\section{Why} 
Case research is particularly useful for "how" and "why" questions 
about current events, 
but where the researcher(s) cannot 
control what happens (ruling out an experimental case-control approach). 

\begin{table*}[ht]
  \centering
  \fontfamily{ppl}\selectfont
  \begin{tabular}{llll}
    \toprule
    Method & Research question & Control of events? & Contemporary events? \\
    \midrule
    Experiment & how, why? & Yes & Yes \\
    Survey & who, what, where, how many, how much? & No & Yes \\
    Archival analysis & who, what, where, how many, how much? & No & Yes/No \\
    History & how, why? & No & No \\
    Case study & how, why? & No & Yes \\
    \bottomrule
  \end{tabular}
  \caption{When to use each research method. Source: \citet{Yin2009aa}.}
  \label{tab:when_where}
  %\zsavepos{pos:normaltab}
\end{table*}

It can be useful for building new theory 
or testing existing theory, 
as even a single case might be enough to disprove 
a hypothesis\footnote{
This red horse disproves the hypothesis that all of the horses are black or brown.}.
The key reason for looking at only a small number of cases is that 
this allows the research(s) to go into great depth so as to better understand 
the complexities, causal links and reasons behind a phenomenon, 
as opposed to random sampling methods where a large number of cases are examined, but only shallowly\cite{Denscombe2007aa}\footnote{Correlation is not causality!}.

Case research, however, is often considered inferior to other methods, for a range of reasons. These include: confusion with other methods or the use of case studies for non-research purposes\footnote{e.g. for teaching or in journalism}; misconceptions that the cases themselves have to be representative\footnote{Sampling approaches are discussed later in this two-pager, but (in short) cases might be selected because they are representative. However, a critical case (that might be non-representative) might be selected instead so as to understand why something happened (e.g. why is this horse red, but all the others are black or brown? Did someone paint it red?).} or that statistical significance is the only way to achieve scientific rigor\cite{Bonoma1984aa,}\marginnote{Eisenhardt \& Graebner (2007); Denscombe (2007), Yin (2009).}


\subsection{The "duality criterion"}
This is that case study research needs to be grounded in the context of the case(s)\footnote{So as to get to the level of detail and depth of findings that can be achieved by only looking at a few cases (i.e. only looking at one field of horses).}, while at the same time seeking generalisation\footnote{i.e. making/seeking findings than can be confidently assumed to apply to all fields of horses.}\footnote{Ketokivi and Choi (2014)}. Denscombe (2007) even suggests that researchers should provide "an explicit defence against the allegation that you cannot generalize from (the) case study findings"\footnote{i.e. I checked that the red horse in my field was not a brown horse painted red, so therefore we can confidently generalize that red horses do actually exist.}. 

\subsection{Sampling approaches}
The case research methodology literature describes a range of theoretical reasons to include a case in your sample including because: it is a leading example; a critical case; a particularly revelatory case; a representative case; a case that can be studied over time (a longitudinal case study)' a particularly unusual case; or together with other cases provides an opposite result, similar results (for replication) or represents the polar extremes of what might happen. Non-theoretical reasons for including a case might include that it is convenient (especially if geographically proximate), is a forced choice (e.g. because of study funding source) or if a good opportunity arises (e.g. data access)\cite{Eisenhardt1989aa, Meredith1998aa, Stuart2002aa, Voss2002aa,}\marginnote{Barratt et al. (2011), Denscombe (2007), Ketokivi and Choi (2014), Yin (2009)}. Such constraints are find, but should be acknowledged. As well, just because something is initially included for non-theoretical reasons doesn't mean that its inclusion might also be justified on theoretical grounds.  

How many cases to include is also an issue.  One or only a few might not provide sufficient breadth.  Too many and it is difficult to get the depth of research necessary to fully understand the context and mechanisms. Eisenhardt (1989) suggests four to ten, Yin (2009) six to ten including two to three specifically chosen for replication and the others chosen to explore differences.  But even one might be enough, especially in theory testing studies\footnote{Again, this one red horse disproves the theory that all horses are brown or black.  No, there is no need for me to find another red horse, one is enough.}.


\section{Is everything in transport research a case study?}
Most transport research is grounded in the context of a place, be it a city, country or other unit of analysis. From a certain viewpoint, perhaps all of our studies are of specific cases.  Even experimental transport modelling might make assumptions about local traffic laws, cultural practices with respect to pedestrian behaviour or queuing, or other elements grounded in the context of the case that is being studied or in which the researcher(s) are themselves located. 

A \emph{key consequence for us} therefore is to recognise the need to seek generalisable findings, that can be confidentially applied to other places and cases.  For example, there was a lot of research activity related to COVID-19 impacts on WFH and travel. Results from surveys of people in (the cases of) Melbourne and Perth may represent polar extremes with respect to the impacts of infection fear and lockdowns.  Can these results be confidently generalised to other places in Australia or even internationally? Perhaps, if we frame the findings from Melbourne and Perth as being indicative of the two extremes that might be reasonably expected in first-world countries, where infection rates were low, vacination rates were high, etc. Are these cases generalisable to everywhere?  Probably not entirely, but they may be indicative of wider trends or patterns.  Even in such quantitative surveys there may be \emph{a need to address the duality criterion}, which may be easier if the fact that you are actually effectively doing a case study (within which there is a quantitative study) is acknowledged.   

The case research methodology literature is a rich resource that is applicable to a lot of different research approaches. It includes recommendations with respect to: framing the study questions, an especially which level questions are asked at (overall study, cross-case, case, sub-unit, individual participant); tools such as case study protocols, case reports and cross-case reports that can ensure (and demonstrate) academic rigor, and help to direct and formalise the study; and a body of theory that supports the use of case studies as a research tool.  


\nobibliography{case_study_methods.bib}
\bibliographystyle{plainnat}



\end{document}
